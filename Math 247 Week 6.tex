\documentclass[a4paper, 12pt]{article}
\usepackage{amsthm}
\setlength{\parindent}{0pt}
\newcommand{\forceindent}{\leavevmode{\parindent=3em\indent}}
\usepackage{fancyhdr}
\usepackage{graphicx}
\usepackage{float}
\newcommand{\eqname}[1]{\tag*{#1}}% Tag equation with name
\usepackage{subfig}
\usepackage{caption}
\usepackage{graphicx}
\usepackage{gensymb}
\usepackage[margin=2cm]{geometry}
\newtheorem{theorem}{Theorem}
\usepackage{layout}
\usepackage{enumitem}
\usepackage{amsmath}
\usepackage{amsmath}
\newtheorem{corollary}{Corollary}[theorem]
\newtheorem{lemma}[theorem]{Lemma}
\usepackage{mathtools}
\usepackage{tikz}
\usepackage{color,soul}
\usepackage{amssymb}
\usepackage[utf8]{inputenc}
\usepackage{csquotes}
\usepackage{color}
\usepackage[TS1,T1]{fontenc}
\usepackage{array, booktabs}
\usepackage{booktabs}
\usepackage{tabularx}
\usepackage{graphicx}
\usepackage{color}
\setlength{\parindent}{0pt}
\usepackage[utf8]{inputenc}
\usepackage{hyperref}
\usepackage{csquotes}
\usepackage{mathrsfs}
\newsavebox\foobox
\newlength{\foodim}
\newcommand{\slantbox}[2][0]{\mbox{%
        \sbox{\foobox}{#2}%
        \foodim=#1\wd\foobox
        \hskip \wd\foobox
        \hskip -0.5\foodim
        \pdfsave
        \pdfsetmatrix{1 0 #1 1}%
        \llap{\usebox{\foobox}}%
        \pdfrestore
        \hskip 0.5\foodim
}}
\def\Laplace{\slantbox[-.45]{$\mathscr{L}$}}
\setlength{\arrayrulewidth}{0.2mm}
\setlength{\tabcolsep}{18pt}
\renewcommand{\arraystretch}{1.0}

\theoremstyle{definition}
\newtheorem{exmp}{Example}[section]

\theoremstyle{definition} 
\newtheorem{defn}{Definition}[section]

\theoremstyle{definition}
\newtheorem{alg}{Algorithm}[section]

\theoremstyle{definition}
\newtheorem{rmk}{Remark}[section]

%TO USE CODE
\usepackage[T1]{fontenc}
\usepackage{inconsolata}

\usepackage{color}

\definecolor{pblue}{rgb}{0.13,0.13,1}
\definecolor{pgreen}{rgb}{0,0.5,0}
\definecolor{pred}{rgb}{0.9,0,0}
\definecolor{pgrey}{rgb}{0.46,0.45,0.48}

\usepackage{listings}
\lstset{language=Java,
	showspaces=false,
	showtabs=false,
	breaklines=true,
	showstringspaces=false,
	breakatwhitespace=true,
	commentstyle=\color{pgreen},
	keywordstyle=\color{pblue},
	stringstyle=\color{pred},
	basicstyle=\ttfamily,
	moredelim=[il][\textcolor{pgrey}]{$ $},
	moredelim=[is][\textcolor{pgrey}]{\%\%}{\%\%}
}





\title{\textbf{Math 247: Applied Linear Algebra} \vspace{-2ex}}
\author{\textbf{Course Notes - Week 6} \vspace{-2ex}}
\date{}


\newenvironment{enumerate_tight}{
	\begin{enumerate}
		\setlength{\itemsep}{0pt}
		\setlength{\parskip}{0pt}
	}{\end{enumerate}}
\newenvironment{itemize_tight}{
	\begin{itemize}
		\setlength{\itemsep}{0pt}
		\setlength{\parskip}{0pt}
	}{\end{itemize}}

\begin{document}

\maketitle
%\tableofcontents 
\textbf{Main topics}: Compositions of linear maps, matrix representations of isomorphisms, change of basis, similarity, normal form problem, equivalence relation. 

\pagestyle{fancy}
\lhead{Math 247: Applied Linear Algebra}
\rhead{}
\rhead{Revision Notes}
\lfoot{}
\cfoot{}
\rfoot{Page \thepage}
\renewcommand{\headrulewidth}{0.4pt}
\renewcommand{\footrulewidth}{0.4pt}
\setlength{\tabcolsep}{0.5em} % for the horizontal padding
{\renewcommand{\arraystretch}{1.2}% for the vertical padding

\section{From Last Time} 
Recall the example from last class:
\begin{align*}
	L: &  Mat(2 \times 2, \mathbb{R}) \rightarrow Mat(2 \times 2, \mathbb{R}) \\
	& A \mapsto A - A^T \\
	& B = \begin{bmatrix}
		1 & 0 \\
		0 & 0 
	\end{bmatrix}, \begin{bmatrix}
		0 & 1 \\
		0 & 0 
	\end{bmatrix}, \begin{bmatrix}
		0 & 0 \\
		0 & 1
	\end{bmatrix}, \begin{bmatrix}
		0 & 0 \\
		0 & 1 
	\end{bmatrix}
\end{align*}
We can then obtain $[L]_{B,B}$ by unravelling the matrices from the basis and using them as the columns of $[L]_{B,B}$
\begin{align*}
	[L]_{B,B} = \begin{bmatrix}
		0 & 0 & 0 & 0 \\
		0 & 1 & -1 & 0 \\
		0 & -1 & 1 & 0 \\
		0 & 0 & 0 & 0 
	\end{bmatrix}
\end{align*}
Consider an alternative basis $B'$. 
\begin{align*}
	L: &  Mat(2 \times 2, \mathbb{R}) \rightarrow Mat(2 \times 2, \mathbb{R}) \\
	& A \mapsto A - A^T \\
	& B' = \begin{bmatrix}
		1 & 0 \\
		0 & 0 
	\end{bmatrix}, \begin{bmatrix}
		0 & 1 \\
		1 & 0 
	\end{bmatrix}, \begin{bmatrix}
		0 & 0 \\
		0 & 1
	\end{bmatrix}, \begin{bmatrix}
		0 & 1 \\
		-1 & 0 
	\end{bmatrix}	
\end{align*}
Applying the transformation, we then obtain: 
\begin{align*}  
	 L(B') = \begin{bmatrix}
		0 & 0 \\
		0 & 0 
	\end{bmatrix}, \begin{bmatrix}
		0 & 0 \\
		0 & 0 
	\end{bmatrix}, \begin{bmatrix}
		0 & 0 \\
		0 & 0
	\end{bmatrix}, \begin{bmatrix}
		0 & -2 \\
		-2 & 0 
	\end{bmatrix}	
\end{align*} 
We need to express the images as linear combinations of $B'$. Trivial for the first three: 
\begin{align*}  
	  = 0 \cdot \begin{bmatrix}
		0 & 0 \\
		0 & 0 
	\end{bmatrix} + 0 \cdot \begin{bmatrix}
		0 & 0 \\
		0 & 0 
	\end{bmatrix} + 0 \cdot \begin{bmatrix}
		0 & 0 \\
		0 & 0
	\end{bmatrix} + 2 \cdot  \begin{bmatrix}
		0 & -1 \\
		-1 & 0 
	\end{bmatrix}
\end{align*} 
Collect coefficients into a $4 \times 4$ matrix: 
\begin{align*}
	\Rightarrow [L]_{B',B'} = \begin{bmatrix}
		0 & 0 & 0 & 0 \\
		0 & 0 & 0 & 0 \\
		0 & 0 & 0 & 0 \\
		0 & 0 & 0 & 2 
	\end{bmatrix} \\
	[L]_{B,B} = \begin{bmatrix}
		0 & 0 & 0 & 0 \\
		0 & 1 & -1 & 0 \\
		0 & -1 & 1 & 0 \\
		0 & 0 & 0 & 0 
	\end{bmatrix}
\end{align*}
\textbf{The normal form question}: what basis can we choose to obtain the simplest matrix? 
\section{Compositions of linear maps}
Let U, V, and W be finite-dimensional VS over k with bases B, B', and B". Let's define two linear maps: 
\begin{align*}
	& L_1: U_B \rightarrow V_{B'} \\
	& L_2: V_{B'} \rightarrow W_{B"} 
\end{align*}
Then, we consider the following transformation: 
\begin{align*}
	L_2 \circ L_1 = U_B \rightarrow W_{B"} 
\end{align*}
\textbf{Question}: how are $[L_1]_{B',B}$, $[L_2]_{B",B'}$, and $[L_2 \circ L_1]_{B", B}$ related? 

\begin{theorem}
	Let $L_1$ and $L_2$ be defined as above. Let $dim(V) = m$, $dim(V) = k$, and $dim(W) = n$. Then: 
	\begin{align*}
		[L_2 \circ L_1 ]_{B", B} = [L_2]_{B",B} \cdot [L_1]_{B',B} 
	\end{align*}
\end{theorem}
This theorem is basically saying that you can obtain a matrix composition by multiplying the matrix maps. This is why matrix multiplication is so complicated; it's defined to be that way such that we can use it for linear maps. 
\begin{proof}
	Let $u$ be an arbitrary vector in U. Then, the idea is to simply compute what $[L_2 \circ L_1]_{B", B} [u]_B = [(L_2 \circ L_1)(u)]_{B"}$ is. As was discussed last week, we know that: 
	\begin{align*}
		[L_2(L_1(u))]_{B"} = [L_2]_{B", B'}[L_1(u)]_{B'}
	\end{align*}
	We are going to apply the same idea but backwards: 
	\begin{align*}
		& \Rightarrow [L_2]_{B", B'} [L_1]_{B',B}[u]_B \\
		& \Rightarrow [L_2 \circ L_1]_{B", B} = \mbox{ product of } [L_2]_{B",B'}[L_1]_{B',B} \\
		& \Rightarrow [L_2 \circ L_1]_{B", B} = [L_2]_{B", B'} \cdot [L_1]_{B', B} 
	\end{align*}
\end{proof}

\section{Matrix Representations of Isomorphisms}
\begin{theorem}
	V,W finite-dimensional VS over k, $L:V \rightarrow W$ be linear and let B, B' be bases for V, W, respectively. Claim: the linear map is an isomorphism $\iff$ the matrix representation of the linear map $[L]_{B',B}$ is invertible. This is  the same thing as insisting that L is bijective. 
\begin{rmk}
	Consider the identity map, $id: V \rightarrow V$, with $dim(V) = n$, and a basis B. Then: 
	\begin{align*}
		[id]_{B,B} = I_n \mbox{ ($n \times n$ identity matrix) } 
	\end{align*}
	Why? We can just use the definition of a matrix map. 
	\newline
	\newline
	If $B = (v_1, ..., v_n)$, then the matrix representation of $[id]_{B,B}$ would then be given by: 
	\begin{align*}
		([id(v_1)]_B|\hdots | [id(v_n)]_B)
	\end{align*}
	From last week's lectures, it therefore follows: 
	\begin{align*}
			=	([v_1]_B|\hdots | [v_n]_B)
	\end{align*}
	By expressing each $v_i$ as a linear combination of $v_1,..., v_n$, we then obtain the following matrix:
	\begin{align*}
		= \begin{bmatrix}
			1 & 0 & \hdots & 0 \\
			0 & 1 & \hdots & \vdots \\
			\vdots & \vdots & \ddots & \vdots \\
			0 & 0 & \hdots & 1
		\end{bmatrix}
		= I_n 
	\end{align*}
\end{rmk}
\end{theorem}
Now, to prove the theorem: 
\begin{proof} Proving both directions: \newline
[ $\Rightarrow$ ] Let L: V (B) $\rightarrow$ W (B') be an isomorphism. This means that L is bijective $\Rightarrow$ automatically has an inverse map that we automatically know is linear: L has an inverse $L^{-1}$ which is $L^{-1} : W_{B'} \rightarrow V_{B}$ We now have: $L^{-1} \circ L: V_B \rightarrow V_B, L^{-1} \circ L = id_V$. 
\newline
\newline
We now apply the remark and composition formula:
\begin{align*}
	[L^{-1} \circ L]_{B,B} = I = [L^{-1}]_{B,B'} \cdot [L]_{B',B} 
\end{align*}
Matrices whose product is I implies that one of them is invertible. Similarly: 
\begin{align*}
	[L]_{B',B} \cdot [L^{-1}] = I \Rightarrow [L]_{B',B} \mbox{ is invertible. } 
\end{align*}
[ $\Leftarrow$ ] We know that our matrix is invertible. So, we need to prove that it is bijective. So, let $[L]_{B',B}$ be an invertible $n \times n$ matrix. We will prove that L is injective first. To prove this, we need to show that the kernel is trivial. 
\newline
\newline
Let $v \in Ker(L) \Rightarrow L(v) = 0 \Rightarrow [L(v)]_{B'} = 0$. We then obtain: 
\begin{align*}
	 [L]_{B', B} \cdot [v]_B = [L(v)]_{B'} = 0 
\end{align*}
We know that $[L]_{B',B}$ is an invertible matrix, so, it cannot be zero. This means that $[v]_B$ must be zero $\Rightarrow$ $v = 0$. We chose an arbitrary vector in the kernel, showed that it was equal to zero. This means that $Ker(L) = \{ 0 \}$, which means that L is injective. 
\newline
\newline
Now, we need to show that L is \textbf{surjective}. Since $[L]_{B',B}$ is an $n \times n$ matrix, we have that: 
\begin{align*}
	dim(V) = dim(W) = n 
\end{align*}
By the dimension formula, we obtain: 
\begin{align*}
	& dim(Ker) + dim(Im) = dim(V)  \\
	& 0 + dim(Im) = n 
\end{align*}
This implies that W is an n-dimensional subspace of an n-dimensional vector space. The only way this is possible is that if $Im(L) = W$, which is the very definition of L being surjective. So, since injectivity and surjectivity hold, we then obtain that L is bijective, and from this we obtain that L is an isomorphism. 
\end{proof}

\section{Change of Basis}
There are two questions that will motivate this section: 
\begin{enumerate}
	\item Let $v \in V$. Given the coordinate vector $[v]_B$ of $v$ with respect to the basis B, how can we compute the coordinate vector $[v]_{B'}$ with respect to B', another basis for the same vector space? 
	\item Let $L: V \rightarrow V$ be linear. Given the matrix representation $[L]_{B,B}$, how can we compute $[L]_{B',B'}$. 
\end{enumerate}
\subsection{Question 1}
Consider the linear map $id: V_B \rightarrow V_{B'}$. The matrix representation is given by $[id]_{B',B}$. By definition, we obtain: 
\begin{align*}
	[id]_{B',B} [v]_B = [id(v)]_{B'} = [v]_{B'} 
\end{align*}
The \textbf{transition matrix/change of basis matrix} from B to B' is given by $[id]_{B',B} \cdot [v]_{B}$. This formula is important: 
\begin{align}
	[v]_{B'} = [id]_{B',B} \cdot [v]_B
\end{align}
\subsection{Question 2}
Consider the following sequence of maps: 
\begin{align*}
	V_{B'} \xrightarrow{id} V_{B} \xrightarrow{L} V_{B} \xrightarrow{id} V_{B'}  
\end{align*}
Then, 
\begin{align*}
	& [L]_{B',B'}= [id \circ L \circ id]_{B',B'} \\
	& \Rightarrow [id]_{B',B} [L]_{B,B} [id]_{B,B'} \\
	& \Rightarrow ([id]_{B,B'})^{-1} [L]_{B,B} [id]_{B,B'} 
\end{align*}
Therefore, $[L]_{B,B}$ and $[L]_{B',B'}$ are similar matrices. 
This is \textbf{how we transform one matrix representation to another}. Representations of the same linear map are possible if there exists an invertible matrix P.
\begin{defn}
	Two $n$ by $n$ matrices A,B with coefficients in $K$ are called \textbf{similar} if there exists an invertible $n \times n$ matrix P with: 
	\begin{align*}
		B = P^{-1} A P
	\end{align*}
	Thus, any two matrix representations of the same linear map $L: V \rightarrow V$ with respect to bases B, B' are similar. 
\end{defn}
\textbf{\underline{Special Case (Most important)}}
\newline
V = $K^n$ with two bases: the standard basis S and an arbitrary basis B. \textbf{First Question:} if we consider the vectors in $K^n$ as row vectors, what is the coordinate vector of $[v]_s$?  The solution: 
\begin{align*}
	[v]_s = v^t
\end{align*}
How can we compute the coordinate vector $[v]_B$? 
\begin{align*}
	[v]_B = [id]_{B,S}[v]_S 
\end{align*}
Where $[id]_{S,B}$ is the transition matrix B to S. Now, we can \textbf{feed in arbitrary members of B} by expressing them in terms of the standard basis. Let $B = (v_1, ..., v_n)$. Then: $[id]_{S,B} = (v_1^t | \cdots | v_n^t)$ and $[id]_{B,S} = ([id]_{S,B})^{-1}$. So, we can basically find the vector in terms of the original basis by ``inverting'' the transformation: 
\begin{align*}
	[v]_B = P^{-1}[v_s] 
\end{align*}
Where $P = (v_1^t| \cdots | v_n^t)$. 
\begin{exmp}
	V = $\mathbb{R}^2$, S standard basis, and $B = ((1,1),(3,2))$. From this, we can obtain two change of basis matrixes: 
	\newline
	\newline
	$[id]_{S,B}$: from B to S. Relatively straightforward; just take the rows of B and use them as the columns of the change of basis matrix: 
	\begin{align*}
		[id]_{S,B} = \begin{bmatrix}
			1 & 3 \\
			1 & 2 
		\end{bmatrix}
	\end{align*}
	$[id]_{B,S}$ this is more complicated; we need to compute $([id]_{S,B})^{-1}$. We then obtain: 
	\begin{align*}
		[id]_{B,S} = \begin{bmatrix}
			-2 & 3 \\
			1 & -1 
		\end{bmatrix}
	\end{align*}
	Putting it all together, we then obtain: 
	\begin{align*}
		[v]_B = [id]_{B,S} = [v]_S
	\end{align*}

\end{exmp}
\begin{exmp}
	$v = (2,3)$. So, $[v]_S = \begin{pmatrix}
		2 \\ 3 
	\end{pmatrix}$. We can obtain $[v]_B$ by: 
	\begin{align*}
		[v]_B = \begin{bmatrix}
			-2 & 3 \\
			1 & -1 \\ 
		\end{bmatrix} \begin{bmatrix}
			2 \\
			3 
		\end{bmatrix} = \begin{bmatrix}
			5 \\
			1
		\end{bmatrix}
	\end{align*}
\end{exmp}
\begin{exmp}
	$L: \mathbb{R}^2 \rightarrow \mathbb{R}^2$. Remark that this is an endomorphism. Let this linear map define the projection onto the x-axis; i.e,: 
	\begin{align*}
		& L((1,0)) = (1,0) \\
		& L((0,1,)) = (0,0) \\
		& \Rightarrow [L]_{S,S}  = \begin{bmatrix}
			1 & 0 \\
			0 & 0 
		\end{bmatrix}
	\end{align*}
	Using the formula, we then obtain: 
	\begin{align*}
		[L]_{B,B} &  = [id]_{B,S} \cdot [L]_{S,S} \cdot [id]_{S,B} \\
		& = ([id]_{S,B})^{-1} [L]_{S,S} [id]_{S,B} \\ 
		& = \begin{bmatrix}
			-2 & 3 \\
			1 & -1 
		\end{bmatrix} \begin{bmatrix}
			1 & 0 \\
			0 & 0  
		\end{bmatrix} \begin{bmatrix}
			1 & 3 \\
			1 & 2 
		\end{bmatrix} = \begin{bmatrix}
			-2 & -6 \\
			1 & 3
		\end{bmatrix}
	\end{align*}
To check if your solution is correct, you feed in a coordinate vector, compute the output, and verify that it is correct. 
\end{exmp}
\begin{exmp}
	V = Mat($2 \times 2, \mathbb{R}$), $L: A \mapsto A - A^t$. Define the following bases: 
	\begin{align*}
		B = \begin{bmatrix}
			1 & 0 \\
			0 & 0 
		\end{bmatrix}, \begin{bmatrix}
			0 & 1 \\
			0 & 0 
		\end{bmatrix}, \begin{bmatrix}
			0 & 0 \\
			1 & 0 
		\end{bmatrix}, \begin{bmatrix}
			0 & 0 \\
			0 & 1 
		\end{bmatrix} \\
		B' = \begin{bmatrix}
			1 & 0 \\
			0 & 0 
		\end{bmatrix}, \begin{bmatrix}
			0 & 1 \\
			1 & 0 
		\end{bmatrix}, \begin{bmatrix}
			0 & 0 \\
			0 & 1 
		\end{bmatrix}, \begin{bmatrix}
			0 & 1 \\
			-1 & 0 
		\end{bmatrix}
	\end{align*}
	Last class, we computed the following matrices: 
	\begin{align*}
		& [L]_{B,B} = \begin{bmatrix}
			0 & 0 & 0 & 0 \\
			0 & 1 & -1 & 0 \\
			0 & -1 & 1 & 0 \\
			0 & 0 & 0 & 0 
		\end{bmatrix}, [L]_{B', B'} = \begin{bmatrix}
			0 & 0 & 0 & 0 \\
			0 & 0 & 0 & 0 \\
			0 & 0 & 0 & 0 \\
			0 & 0 & 0 & 2
		\end{bmatrix}  \\
		& [id]_{B,B'} = \begin{bmatrix}
			1 & 0 & 0 & 0 \\
			0 & 1 & 0 & 1 \\
			0 & 1 & 0 & -1 \\
			0 & 0 & 1 & 0 
		\end{bmatrix}, [id]_{B',B} = ([id]_{B,B'})^{-1} = \frac{1}{2} \begin{bmatrix}
			2 & 0 & 0 & 0 \\
			0 & 1 & 0 & 0 \\
			0 & 0 & 0 & 2 \\
			0 & 1 & -1 & 0 
		\end{bmatrix}
	\end{align*}
\end{exmp}
\section{The Normal Form Problem}
Similar matrices represent the same linear map with respect to different bases. So, let $A \in MAT(n \times n, k)$. Let $L_A: K^n \rightarrow k^n$, $v \mapsto Av$. Let S be a standard basis for $K^n$ be an arbitrary basis. Then, $[L_A]_{S,S} = A, B = \{ v_1, ..., v_n \}$. Then: 
\begin{align*}
	[L_A]_{S,S}&  = [id]_{B,S} [L_A]_{S,S} [id]_{S,B} \\
				& =  P^{-1} A P 
\end{align*}
Where the columns of P consist of the vectors in B written as column vectors: $P = (v_1^t | \cdots | v_n^t)$. P is invertible $\Rightarrow$ we have n linearly independent columns in $K^n$ $\Rightarrow$ they span $K^n$ $\Rightarrow$ they form a basis for $K^n$. 
\newline
\textbf{\underline{Algebra Crash Course}}
\newline 
\textbf{General linear group}: denoted $P \in GL(n,k)$, a general linear group is the set of all invertible $n \times n$ matrices with coefficients in k. It's a group. 
\newline
\newline
\textbf{Question:} can we interpret $P^{-1}AP$ as the matrix representation of our linear map L with respect to some basis B of $K^N$? Yes! We can just take $P = (v_1^t | \cdots | v_n^t)$. Here's the chain of reasoning: P is invertible $\Rightarrow$ $(v_1, ..., v_n)$ are linearly independent; $dim(K^n) = n$ $\Rightarrow$ $\{ v_1, ..., v_n \}$ also span $K^n$ $\Rightarrow$ basis for $K^n$. So, we can write: 
\begin{align*}
	[L_A] = P^{-1}AP
\end{align*}
We will see that there is a one-to-one correspondence between matrix representations of our linear map L and matrices similar to A. In other words: \textbf{matrices similar to A can be interpreted as representing the same linear map, just with respect to a different basis}. So, in the Linear Algebra Universe, they're practically equivalent. This brings us to the final section.
\section{Similarity Problem}
\textbf{Motivation:} we are interested in knowing the set of all matrices $\in$ MAT $(n \times n, k)$ similar to A. This is the \textbf{similarity problem}. This introduces a relation: a set of any two matrices can be similar or not. We obtain two mathematical relations from this: 
\begin{enumerate}
	\item Order
	\item Class of equivalence
\end{enumerate}
\begin{defn}
	Let $A, B \in MAT (n \times m, k)$. B is said to be \textbf{similar} to A, in symbols, $B \sim A$, if there exits a $P \in GL(n,k)$ such that
	\begin{align*}
		B = P^{-1} A P 
	\end{align*}
\end{defn}
Before this, we need more quick algebra. 
\subsection{Quick Algebra}
\begin{defn}
	Let S be a set. An \textbf{equivalence relation} is a subset of the cartesian product, $S \times S$. In maths, it's denoted by a $\sim$ and has the following three properties: 
	\begin{enumerate}
		\item \textbf{Identity:} every element of S is in relation with itself: 
		\begin{align*}
			\forall x \in S, x \sim s
		\end{align*}

	\item \textbf{Commutativity}: 
	\begin{align*}
		\forall x, y \in S, \mbox{ if } x \sim y, \mbox{ then } y \sim x \\
		\mbox{ Alternatively: } x \sim y \iff y \sim x
	\end{align*}
	\item \textbf{Transitivity:} 
	\begin{align*}
		\forall x, y, z \in S, \mbox{ if } x \sim y \land y \sim z \Rightarrow x \sim z
	\end{align*}
	\end{enumerate}
\end{defn}
\begin{exmp}
	$S = \mathbb{Z}, n \sim k \iff n-k$ is even. \textbf{Claim:} this is an equivalence relation. We can check this with the axioms. 
	\begin{enumerate}
		\item $n \sim n$ since $n-n =0$ which is even. 
		\item $n \sim k \Rightarrow n-k$ is even $\iff k-n$ is even $\Rightarrow k \sim n$. 
		\item $n \sim k \land k \sim m \Rightarrow$ show that $n-m$ is also even: $(n-k) + (k-m) \Rightarrow$ we have the sum of two even things, so the sum will also be even and we have $n \sim m$, as desired. 
	\end{enumerate}
\end{exmp}
\textbf{\underline{Main idea:} equivalence relationships partition things into classes. In this case, all even numbers are similar/equivalent to zero and all odd numbers are equivalent to 1. So, we have the set of all integers partitioned into two, disjoint groups: (1) odd numbers, and (2) even numbers.}
\begin{align*}
	\mbox{ The set of all } \{ n \in Z\  n \sim 0 \} = \{ ...,-4,-2,0,2,4,...\} \mbox{ (the set of all even numbers) } \\
		\mbox{ The set of all } \{ n \in Z\  n \sim 1 \} = \{ ...,-3,-1,1,3,...\} \mbox{ (the set of all odd numbers) } 
\end{align*}
\begin{defn}
	Let S be a set, $s \in S, \sim$ an equivalence relation on S. Then, we can define the \textbf{equivalence class} of S as follows: 
	\begin{align*}
		[s]_{\sim} := \{ t \in S\ |\ t \sim S \} 
	\end{align*}
\end{defn}
Remark that $s \in [S]_{\sim}$ by axiom 1 of equivalence relations. Back to our example: 
\begin{exmp}
	$S \in \mathbb{Z}, \sim, n-k$ even. Then: 
	\begin{enumerate}
		\item $[0]$ is the set of all even numbers. 
		\item $[1]$ is the set of all odd numbers. 
	\end{enumerate}
	These are equivalence classes that \textbf{partition} $\mathbb{Z}$ (the set containing all integers). This, in fact, is true in general: 
	\begin{theorem}
		Let S be a set, $\sim$ be an equivalence relation. The equivalence classes of $\sim$ \textbf{partition} S, i.e., every $s \in S$ is contained in an equivalence class and any two equivalence classes are either \textbf{identical} or \textbf{disjoint}. 
	\end{theorem}
Proof is next class...
\end{exmp}
\end{document}